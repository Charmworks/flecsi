% ENAMR-RDIT Requirements section.
\section{Requirements}\label{RequirementsSec}

Table \ref{amr_req_table} gives the AMR requirements for ENAMR; Table \ref{data_req_table} gives the data requirements.


\newcounter{reqTable}
\setcounter{reqTable}{1}
\newcounter{reqNum}[reqTable]
% \numberwithin{reqNum}{table}
\newcommand\reqNumber{\stepcounter{reqNum}\thereqTable.\arabic{reqNum}}

\begin{table}[hbt]
%\centering
\begin{tabular}{c p{5 in}}
\toprule
Requirement & \\
\midrule
\reqNumber & Associate data fields with homogeneous sets of elements. \\
& \emph{Note 1: a field is associated with only one type of element.} \\
& \emph{Note 2: a field may be associated with any subset of the elements of a mesh, including all of the elements} \\
& \emph{Note 2: a field need only be associated with the elements of one mesh.} \\
\reqNumber & Provide data fields associated with cell centers.\\
\reqNumber & Provide data fields associated with cell edges.\\
\reqNumber & Provide data fields associated with cell faces.\\
\reqNumber & Provide data fields associated with cell vertices.\\
\reqNumber & Provide data fields associated with cell corners.\\
\reqNumber & Provide data fields associated with cell wedges.\\
\reqNumber & Associate data fields with materials.
& \emph{Note 1: a field can be associated with 0 or more materials.} \\
\reqNumber & Associate data fields with isotopes.
& \emph{Note 1: a field can be associated with 0 or more isotopes.} \\
\reqNumber

\reqNumber & Data Store: manage data fields: arrays associated with tiles. \\
\reqNumber & Provide ability to index data fields in data store.\\
& \emph{For example, index by tile, material, and another key such as field name. That is, distinguish between ``temperature, material 1, tile 1'' and ``temperature, material 2, tile 1''.}\\
\bottomrule
\end{tabular}
\parbox{5in}{\caption{Requirements for Data Fields}\label{data_req_table}}
\stepcounter{reqTable}
\end{table}


\clearpage

