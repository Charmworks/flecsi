%  FleCSI Data Requirements: Field requirementes section.

\section{Field Requirements}\label{FieldRequirementsSec}

\subsection{Use Cases}

\begin{table}[hbt]
\begin{tabular}{c p{5 in}}
\toprule
Use Case & Description \\
\midrule
\useNumber & adds a field on request: adds metadata, associates metadata to field\\
\useNumber & allocates the FieldData corresponding to the field in the right memory\\
\useNumber & provides accessors to field data\\
\useNumber & deletes a field\\
\useNumber & deletes FeildData\\
\useNumber & retrieves a field\\
\useNumber & searches fields for matching metadata\\
\bottomrule
\end{tabular}
\centering
\parbox{5in}{\caption{Use cases for DataStore\label{data_store_field_use_case_table}}}
\stepcounter{useTable}
\end{table}


\begin{table}[hbt]
\begin{tabular}{c p{5 in}}
\toprule
Use Case & Description \\
\midrule
\useNumber & Is given read or write access to a field\\
\useNumber & updates a field\\
\useNumber & can declare or initialize a field\\
\useNumber & associates a field with an index set (dense or sparse) \\
\bottomrule
\end{tabular}
\centering
\parbox{5in}{\caption{Task/Driver use cases for Fields.\label{task_field_use_case_table}}}
\stepcounter{useTable}
\end{table}


\subsection{Field Requirements}

% \begin{table*}[!hbt]
% \begin{tabular}{c p{5 in}}
\begin{longtable}{c p{5 in}}
\caption[]{Requirements for Fields\label{data_field_req_table}}\\
\toprule
Requirement & Description \\
\midrule
\endfirsthead
\caption[]{(cont.) Requirements for Fields}\\
\toprule
Requirement & Description \\
\midrule
\endhead
\bottomrule[1.2pt]
\endfoot
\bottomrule[1.2pt]
\endlastfoot
% Association with metadata
& \emph{Metadata} \\
% name
\reqNumber & Associate a name with every Field.\\
& \emph{\tab[0.5cm]Discussion: Does the name need to be unique? Currently, the name is a unique identifier. It's not clear that that needs to be the case, though. For example, can there be a density field for all materials combined (hydro view), and a density field associated with chocolate, peanut butter, etc (chemistry view)? But if there are multiple fields with the same name, how does two packages refer to the same field? Namespacing?}\\
% namespace
\reqNumber & Associate a namespace with every Field.\\
& \Note{1}{The namespace can be defaulted.}\\
& \Note{2}{A name can occur zero or once in a namespace.}\\
% version
\reqNumber & Associate a version with each field.\\
& \Note{1}{This may be required for techniques like predictor-corrector methods.}\\
& \Note{2}{The version distinguishes between multiple instances of the same field name.}\\
& \Example{'pressure after time step 42' versus 'pressure after time step 43'.}\\
& \Discussion{Name, namespace, and version uniquely identify a field.} \\
% Id
\reqNumber & Associate a unique integer identifier with each field.\\
& \Example{We want to be able to talk about a data object independently of its name.}\\
& \Note{1}{an id would be more convenient than specifying name, namespace, and version.}\\
% builtin metadata
\reqNumber & Associate Fields with various builtin metadata aspects.\\
& \Note{1}{a field can be associated with 0 or more materials, 0 or more isotopes.} \\
% user-defined metadata
\reqNumber & Permit users to register metadata in the form of $<$Key,Value$>$ pairs.\\
& \Note{1}{Also permit users to search for and set user--define $<$Key,Value$>$ metadata.}\\
% Search, set
\reqNumber & Search for Fields matching a given metadata criterion.\\
\reqNumber & Set each metadata attribute for each Field.\\
\midrule

% Association of fields & topological elements
& \emph{Topological Elements} \\
\reqNumber & Associate Fields with homogeneous sets of elements of one mesh. \\
& \Note{1}{``homogeneous": only one type of element in the set.} \\
& \Note{2}{a field may be associated with any subset of the elements of a mesh, including all of the elements of that type on the mesh;} \\
& \Note{3}{a field need only be associated with the elements of one mesh.} \\
& \Note{4}{Arbitrary mesh entities, including cell centers, edges, faces, vertices, corners, wedges.}\\

% Association of fields and data
\midrule
& \emph{Data Elements} \\
\reqNumber & Associate fields with FieldData, i.e. the actual data.\\
& \Note{1}{There is a bijective relationship between Fields and FieldDatas.}\\
& \Note{2}{The type of FieldData is fairly arbitrary; it probably includes any type with a null constructor---any type that can be allocated via a new [] statement. Further specification is required.}\\
\reqNumber & Given a field descriptor, get read and write access to the FieldData.\\
& \Note{1}{Read access may be separated from write access.}\\
& \Note{2}{It would be nice to have a way of saying "done writing this field".}\\
\reqNumber & Associate each Field with an IndexSet.\\
& \Note{1}{Exactly one IndexSet per Field.}\\
% & \emph{}
% Initialization and persistence requirements



% Tracking

% \end{tabular}
\stepcounter{reqTable}
\end{longtable}
% \centering
%\parbox{5in}{\caption{Requirements for Fields}\label{data_field_req_table}}
% \end{table*}

Another key concept is the index set with which Field is associated.
The index set defines the subset of mesh entities to which the FieldData correspond.


% \clearpage
\subsection{FieldData}

FieldData are the actual variables associated with the mesh elements.
They represent physically or computationally interesting quantities.
The representation of these data is closely bound up with effecient iteration over them.
Requirements on these data need to be sensitive to the possibility of needing different views and different layouts of the data for different machines and different algorithms.

\begin{table}[hbt]
\begin{tabular}{c p{5 in}}
\toprule
Requirement & Description \\
\midrule
\reqNumber & This table should capture requirements for FieldData. \\
\bottomrule
\end{tabular}
\centering
\parbox{5in}{\caption{Requirements for accessing FieldData, getting to it, allocation it, initializing and persisting it.}\label{field_data_req_table}}
\stepcounter{reqTable}
\end{table}


