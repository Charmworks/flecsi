%  FleCSI Data Requirements: particle requirementes section.

\section{Particle Requirements}\label{ParticleRequirementsSec}

Fields acquire strcuture naturally from the (sub)meshes with which they are associated.
Particles are not always so easily structured.
This requires care and maintenance in specifying requirements for particles.
We assume there will be some kind of organizing principle above the level of individual particles.
Thus we distinguish between:
\begin{enumerate}
\item ParticleSet, a descriptor of a set of particles,
\item ParticleData, space for recording the attributes of particles,
\item ParticleIndex, a ``virtual'' grouping of particles.
\end{enumerate}
\subsection{Use Cases}



\begin{table}[hbt]
\begin{tabular}{c p{5 in}}
\toprule
Use Case & Description \\
\midrule
\useNumber & adds a ParticleSet on request: adds metadata, associates metadata to particle\\
\useNumber & allocates the ParticleData corresponding to the particle in the right memory\\
\useNumber & provides accessors to particle data\\
\useNumber & deletes a particle set\\
\useNumber & deletes particle data\\
\useNumber & retrieves a ParticleSet\\
\useNumber & retrieves a ParticleData\\
\useNumber & searches ParticleSets for matching metadata\\
\bottomrule
\end{tabular}
\centering
\parbox{5in}{\caption{DataStore use cases for particles.\label{data_store_particle_use_case_table}}}
\stepcounter{useTable}
\end{table}


\begin{table}[hbt]
\begin{tabular}{c p{5 in}}
\toprule
Use Case & Description \\
\midrule
\useNumber & Is given read or write access to ParticleData\\
\useNumber & updates ParticleData\\
\useNumber & can declare or initialize a ParticleSet\\
\useNumber & can declare or initialize ParticleData\\
\useNumber & can declare or initialize ParticleIndex\\
\bottomrule
\end{tabular}
\centering
\parbox{5in}{\caption{Task/Driver use cases for particles.\label{task_particle_use_case_table}}}
\stepcounter{useTable}
\end{table}


\subsection{ParticleSet Requirements}

% % \begin{table*}[!hbt]
% % \begin{tabular}{c p{5 in}}
% \begin{longtable}{c p{5 in}}
% \caption[]{Requirements for Particles\label{data_particle_req_table}}\\
% \toprule
% Requirement & Description \\
% \midrule
% \endfirsthead
% \caption[]{(cont.) Requirements for Particles}\\
% \toprule
% Requirement & Description \\
% \midrule
% \endhead
% \bottomrule[1.2pt]
% \endfoot
% \bottomrule[1.2pt]
% \endlastfoot
% % Association with metadata
% & \emph{Metadata} \\
% % name
% \reqNumber & Associate a name with every Particle.\\
% & \emph{\tab[0.5cm]Discussion: Does the name need to be unique? Currently, the name is a unique identifier. It's not clear that that needs to be the case, though. For example, can there be a density particle for all materials combined (hydro view), and a density particle associated with chocolate, peanut butter, etc (chemistry view)? But if there are multiple particles with the same name, how does two packages refer to the same particle? Namespacing?}\\
% % namespace
% \reqNumber & Associate a namespace with every Particle.\\
% & \Note{1}{The namespace can be defaulted.}\\
% & \Note{2}{A name can occur zero or once in a namespace.}\\
% % version
% \reqNumber & Associate a version with each particle.\\
% & \Note{1}{This may be required for techniques like predictor-corrector methods.}\\
% & \Note{2}{The version distinguishes between multiple instances of the same particle name.}\\
% & \Example{'pressure after time step 42' versus 'pressure after time step 43'.}\\
% & \Discussion{Name, namespace, and version uniquely identify a particle.} \\
% % Id
% \reqNumber & Associate a unique integer identifier with each particle.\\
% & \Example{We want to be able to talk about a data object independently of its name.}\\
% & \Note{1}{an id would be more convenient than specifying name, namespace, and version.}\\
% % builtin metadata
% \reqNumber & Associate Particles with various builtin metadata aspects.\\
% & \Note{1}{a particle can be associated with 0 or more materials, 0 or more isotopes.} \\
% % user-defined metadata
% \reqNumber & Permit users to register metadata in the form of $<$Key,Value$>$ pairs.\\
% & \Note{1}{Also permit users to search for and set user--define $<$Key,Value$>$ metadata.}\\
% % Search, set
% \reqNumber & Search for Particles matching a given metadata criterion.\\
% \reqNumber & Set each metadata attribute for each Particle.\\
% \midrule

% % Association of particles & topological elements
% & \emph{Topological Elements} \\
% \reqNumber & Associate Particles with homogeneous sets of elements of one mesh. \\
% & \Note{1}{``homogeneous": only one type of element in the set.} \\
% & \Note{2}{a particle may be associated with any subset of the elements of a mesh, including all of the elements of that type on the mesh;} \\
% & \Note{3}{a particle need only be associated with the elements of one mesh.} \\
% & \Note{4}{Arbitrary mesh entities, including cell centers, edges, faces, vertices, corners, wedges.}\\

% % Association of particles and data
% \midrule
% & \emph{Data Elements} \\
% \reqNumber & Associate particles with ParticleData, i.e. the actual data.\\
% & \Note{1}{There is a bijective relationship between Particles and ParticleDatas.}\\
% & \Note{2}{The type of ParticleData is fairly arbitrary; it probably includes any type with a null constructor---any type that can be allocated via a new [] statement. Further specification is required.}\\
% \reqNumber & Given a particle descriptor, get read and write access to the ParticleData.\\
% & \Note{1}{Read access may be separated from write access.}\\
% & \Note{2}{It would be nice to have a way of saying "done writing this particle".}\\
% \reqNumber & Associate each Particle with an IndexSet.\\
% & \Note{1}{Exactly one IndexSet per Particle.}\\
% % & \emph{}
% % Initialization and persistence requirements



% % Tracking

% % \end{tabular}
% \stepcounter{reqTable}
% \end{longtable}
% % \centering
% %\parbox{5in}{\caption{Requirements for Particles}\label{data_particle_req_table}}
% % \end{table*}

% Another key concept is the index set with which Particle is associated.
% The index set defines the subset of mesh entities to which the ParticleData correspond.


% \clearpage
\subsection{ParticleData}

% ParticleData are the actual variables associated with the mesh elements.
% They represent physically or computationally interesting quantities.
% The representation of these data is closely bound up with effecient iteration % over them.
% Requirements on these data need to be sensitive to the possibility of needing % different views and different layouts of the data for different machines and % different algorithms.
%
% \begin{table}[hbt]
% \begin{tabular}{c p{5 in}}
% \toprule
% Requirement & Description \\
% \midrule
% \reqNumber & This table should capture requirements for ParticleData. \\
% \bottomrule
% \end{tabular}
% \centering
% \parbox{5in}{\caption{Requirements for accessing ParticleData, getting to it, % allocation it, initializing and persisting it.}\label{particle_data_req_table}}
% \stepcounter{reqTable}
% \end{table}


